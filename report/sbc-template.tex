\documentclass[12pt]{article}

\usepackage{sbc-template}


%\usepackage[latin1]{inputenc}  
%\usepackage[utf8]{inputenc}  

\usepackage{graphicx,url}
\usepackage[brazil]{babel}
\usepackage[T1]{fontenc}
\usepackage{amsmath}
\usepackage{listings}
\usepackage{float}

\usepackage{booktabs}
\usepackage{multirow}
\usepackage{siunitx}

\lstset{
  %frame=top,frame=bottom,
  %basicstyle=\small\normalfont\sffamily,    % the size of the fonts that are used for the code
  stepnumber=1,                           % the step between two line-numbers. If it is 1 each line will be numbered
  numbersep=10pt,                         % how far the line-numbers are from the code
  tabsize=2,                              % tab size in blank spaces
  extendedchars=true,                     %
  breaklines=true,                        % sets automatic line breaking
  captionpos=t,                           % sets the caption-position to top
  mathescape=true,
  stringstyle=\color{white}\ttfamily, % Farbe der String
  showspaces=false,           % Leerzeichen anzeigen ?
  showtabs=false,             % Tabs anzeigen ?
  %numbers=left,
  %framexleftmargin=30pt,
  %framexrightmargin=17pt,
  %framexbottommargin=5pt,
  %framextopmargin=5pt,
  showstringspaces=false      % Leerzeichen in Strings anzeigen ?
 }

\renewcommand\lstlistingname{Lista}

\sloppy

\title{IAR0001 - 2017/1\\Relat�rio Trabalho 3\\Algoritmos de Busca}

\author{Alexandre Maros\inst{1} }

\address{Departamento de Ci�ncia da Computa��o -- Universidade do Estado de Santa Catarina\\
  Centro de Ci�ncias Tecnol�gicas -- Joinville -- SC -- Brasil
  \email{alehstk@gmail.com}
}

\begin{document} 

\maketitle

%\begin{abstract}
    %Abstract
%\end{abstract}
     
\begin{resumo} 
    Resumo
\end{resumo}

% 1. Introdu��o
%   Contextualiza��o do problema, justificativa, objetivos, estrutura do relat�rio.
\section{Introdu��o}

Intro

% 2. Problem�tica
%   Detalhamento do problema, PEAS e caracter�sticas do problema
\section{Problem�tica}

Problem�tica

% 3. Modelo implementado
%   Estrat�gias utilizadas, f�rmulas, defini��es de implementa��o, linguagem
\section{Modelo implementado}

O trabalho foi implementado utilizando a linguagem C++ e a biblioteca gr�fica
SFML (\textit{Simple and Fast Multimedia Library}).

Modelo

% 4. Experimentos, resultados e an�lises
%   Detalhamento de como os experimentos foram conduzidos (varia��es do raio, defini��es do n�mero de itens e tamanho da matriz, quantidade de agentes, n�mero de itera��es)
%   Mostrar e analisar os resultados (tabelas, figuras, gr�ficos).
\section{Experimentos, resultados e an�lises}

Experimentos

\begin{table}[h]
    \caption{Resultados obtidos na execu��o de 300 pontos aleat�rios}\label{tab:tab1}
    \centering
    \begin{tabular}{lSSSSSS}
        \toprule
        \multirow{2}{*}{Busca} &
            \multicolumn{2}{c}{Expans�o de N�s} &
            \multicolumn{2}{c}{Custo} \\
            & {$\bar{x}$} & {$\sigma$} & {$\bar{x}$} & {$\sigma$} \\
            \midrule
        Busca em Largura            & 849.24  & 508.25  & 99.89  & 61.37  \\
        Custo Uniforme              & 1848.30 & 1047.06 & 46.41  & 23.84  \\
        Aprofundamente Iterativo    & 4040.81 & 5639.77 & 107.77 & 64.31  \\
        A* ($h(x)$ com peso 1.0)    & 798.53  & 713.67  & 42.37  & 21.56  \\
        A* ($h(x)$ com peso 1.5)    & 620.85  & 592.47  & 45.67  & 21.51  \\
        A* ($h(x)$ com peso 3.0)    & 445.23  & 449.39  & 51.87  & 25.11  \\
        A* ($h(x)$ com peso 5.0)    & 361.74  & 345.95  & 61.22  & 29.64  \\
        \bottomrule
  \end{tabular}
\end{table}

% 5. Conclus�o
%   Considera��es sobre o trabalho e sobre os resultados obtidos, trabalhos futuros.
\section{Conclus�o}

Conclus�o

\bibliographystyle{sbc}
\bibliography{sbc-template}

\end{document}
